\documentclass[12pt,a4paper,titlepage]{article}

\usepackage[utf8x]{inputenc}
\usepackage{ucs}
\usepackage[portuguese]{babel}
\usepackage[T1]{fontenc}
\usepackage{amsmath}
\usepackage{amsfonts}
\usepackage{amssymb}
\usepackage{graphicx}
\usepackage{fancyhdr}
\usepackage{lastpage}
\usepackage{geometry}
\usepackage{float}
\usepackage{makecell}
\usepackage{multirow}
%\usepackage{hyperref}
\usepackage[table,xcdraw]{xcolor}
%\usepackage{caption}
%\usepackage{subcaption}
%\usepackege[nogin]{sweave}



\usepackage{Sweave}
\begin{document}
\Sconcordance{concordance:TrabalhoIntermediario.tex:TrabalhoIntermediario.Rnw:%
1 24 1 1 0 11 1 1 18 6 1 6 0 1 5 6 1 1 2 1 0 2 1 1 2 1 4 3 0 1 3 1 0 1 %
4 3 0 1 2 1 3 2 0 3 2 1 1 3 0 1 2 7 1 1 3 1 0 1 1 4 0 1 3 6 1 1 3 1 0 1 %
1 4 0 1 3 4 1 1 3 1 0 2 1 1 2 1 1 1 2 3 1 1 2 7 1 1 16 18 0 1 3 4 1}


\begin{center}
{\huge Trabalho Intermediário}

{\large Victor Marcius Magalhães Pinto}
\end{center}

\section{Descrição da Tarefa}

teste


\section{Implementação de funções intermediárias}

Para a execução da tarefa, algumas funções auxiliares foram implementadas. A implementação das mesmas pode ser visto a seguir.

\begin{Schunk}
\begin{Sinput}
> 
> #test
> 
\end{Sinput}
\end{Schunk}

\section{Execução do código}

Primeiramente, foram carragadas as imagens contidas no pacote de faces. O pacote contém 40 classes de imagens, com 10 amostras cada.

\subsection{Carregando os dados}

\begin{Schunk}
\begin{Sinput}
> data( faces )
> faces <- t( faces )
> rotate <- function(x) t( apply(x, 2, rev) )
> y <- NULL
> for(i in 1:nrow(faces) )
+ {
+     y <- c( y, ((i-1) %/% 10) + 1 )
+ }
> # Nomeando os atributos
> nomeColunas <- NULL
> for(i in 1:ncol(faces) )
+ {
+     nomeColunas <- c(nomeColunas, paste("a", as.character(i), sep=".") )
+ }
> nomeLinhas <- NULL
> for(i in 1:nrow(faces)) {
+     nomeLinhas <- c(nomeLinhas, paste("face", as.character(y[i]),as.character(i),sep="."))
+ }
> colnames(faces) <- nomeColunas
> faces <- as.data.frame(faces, row.names = nomeLinhas)
> rm(nomeColunas)
> rm(nomeLinhas)
\end{Sinput}
\end{Schunk}

Cada imagem foi transformada, de uma matrix de 64x64 pixels, em um vetor de 4096 features, e inserida em um dataframe de 400 linhas.

\subsection{Diminuindo o número de atributos com PCA}

Em seguida, foi realizada a diminuição do número de atributos da base de dados utilizando o algoritmo do PCA. Em implementações anteriores, foi utilizada a função \textit{preProcess} juntamente com a \textit{predict} do pacote \textit{caret}, o que nos informva o número mínimo de atributos para uma exatidão de 95\% de 123 features. Porém, o pacote apresentou problemas de funcionamento em dias posteriores, o que nos exigiu a utilização da função \textit{prcomp}, que pode ser vista no trecho de código a serguir.


\begin{Schunk}
\begin{Sinput}
> facesPCAaux <- prcomp(faces, center=TRUE, retx=TRUE, scale=TRUE)
> facesPCA <- facesPCAaux$x[,1:5]
> 
\end{Sinput}
\end{Schunk}

Observando a implementação da função do classificador de Bayes, e da geração de pdfs para amostras multivariáveis, uma matriz de correlação invertível é gerada apenas se o número de linhas for menor ou igual ao número de colunas, única possibilidade que não gera uma matriz singular quando da execução da função \textit{solve}. Como são 10 amostras de cada classe, e para o trabalho será utilizada 5 amostras para treino e 5 para testes, o número de features máximo que pode ser utilizado, para o correto funcionamento do algoritmo do classificador de Bayes, é de 5 features.

\subsection{Diminuindo o número de atributos com MDS}

O mesmo procedimento, e a mesma limitação do número de features foi realizado para o me´todo MDS. O algoritmo pode ser visto a seguir.

\begin{Schunk}
\begin{Sinput}
> kMDS <- 5
> facesMDS <- cmdscale(dist(faces), k=kMDS)
> 
\end{Sinput}
\end{Schunk}

\subsection{Gerando sets de treino e teste}

Como dito anteriormente, para o treinamento e classificação pelos algoritmos, foram utilizadas 5 amostras de cada face. A geração das amostras pdoe ser vista no trecho de código a seguir. A mesma possui uma alta complexidade de implementação, visto da utilizazão de loops internos, e de toda o algoritmo, é o trecho que consome mais tempo de execução

\begin{Schunk}
\begin{Sinput}
> dim_classe <- 10
> numClasses <- 400
> numAmostras <- 10
> seqN <- sample(numAmostras)
> porcAmostTrain <- 0.5
> N <- seqN[1:(porcAmostTrain*numAmostras)]
> nSamplesTrain <- length(N)
> n <- seqN[(porcAmostTrain*numAmostras+1):numAmostras]
> nSamplesTest <- length(n)
> xtreino <- c()
> xtreinoPCA <- c()
> xtreinoMDS <- c()
> ytreino <- c()
> xteste <- c()
> xtestePCA <- c()
> xtesteMDS <- c()
> yteste <- c()
> for(r in seq(1,numClasses,numAmostras)) {
+     for(i in N) {
+         xtreino <- rbind(xtreino, (faces[r+i-1,]))
+         xtreinoPCA <- rbind(xtreinoPCA, (facesPCA[r+i-1,]))
+         xtreinoMDS <- rbind(xtreinoMDS, (facesMDS[r+i-1,]))
+         ytreino <- c(ytreino,(y[r+i-1]))
+     }
+     
+     for(i in n) {
+         xteste <- rbind(xteste, (faces[r+i-1,]))
+         xtestePCA <- rbind(xtestePCA, (facesPCA[r+i-1,]))
+         xtesteMDS <- rbind(xtesteMDS, (facesMDS[r+i-1,]))
+         yteste <- c(yteste,(y[r+i-1]))
+     }
+ }
> 
\end{Sinput}
\end{Schunk}




\end{document}
